% Thesis Abstract ------------------------------------------------------

\prefacesection{Abstract}
\vskip -1 cm \vspace{-2mm} 

Particle colliders have produces amazing discoveries in the field of high energy physics during the last years. An important component on this colliders are the particle detectors which identity the particles produced in the collisions. Neutral pion detection is an universal problem in high energy physics. Neutral pion are hard to detect because they decay into two photons with a small angle which are hard to differentiate from one high energy photon. In order to solve this problem researchers from the Science and Technology Center of Valpara�so propose a preshower calorimeter detector, specially designed to have high resolution to identify particles close to each other. The preshower detector is different to the rest of detectors presented on the literature. The read out system is a novel proposal which make the detector different from commonly used calorimeter detectors.

The work of particle detection must be completed using a reconstruction algorithm. The reconstruction algorithm turn the raw data obtained by the detector in useful information about the particle, such as incident position, energy, incident angle, among others. In this work we present the design and implementation of a reconstruction algorithm for the preshower detector. The different literature options are studied and new solutions are proposed for the reconstruction in the preshower detector. Moreover, using computer simulations we analize the design of the detector and we measure its efficacy in the task of detection of particles close to each other, specifically neutral pions.

The algorithm presented is able to detect particles close to each other and reconstruct it position, improving the results obtained by other detectors found in the literature. Also, based on this analysis we propose some possible improvements to the preshower design that can be implemented in new iterations of the detector design.

{\bf Keywords: Reconstruction algorithm, High energy physics. }

% ----------------------------------------------------------------------
